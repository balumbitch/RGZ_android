\chapter*{Заключение}
\addcontentsline{toc}{chapter}{Заключение}

В рамках данной расчетно-графической работы было разработано мобильное приложение MP3-плеер на языке программирования Kotlin с использованием фреймворка Android. Основной функционал приложения включает в себя воспроизведение музыкальных файлов из хранилища устройства, управление треками (воспроизведение, пауза, переход к следующему и предыдущему треку), регулировку громкости и активацию режима цикличного воспроизведения.

Приложение успешно реализует задачи, поставленные в техническом задании:  
1. Загрузка музыкальных файлов из директории \texttt{/Music} и медиа-хранилища устройства.  
2. Управление воспроизведением с помощью кнопок Play/Pause, Next и Back.  
3. Отображение информации о текущем треке и его состоянии на SeekBar.  
4. Реализация простого и интуитивно понятного интерфейса на основе ConstraintLayout.  
5. Обработка системных разрешений для доступа к аудиофайлам.  

В ходе выполнения работы изучены основные компоненты Android SDK для работы с мультимедиа, такие как \texttt{MediaPlayer} и \texttt{AudioManager}. Также были применены базовые подходы к организации пользовательского интерфейса и управления разрешениями.

Основным направлением для дальнейшего улучшения приложения может стать реализация плейлиста с возможностью выбора треков пользователем, создание уведомлений о текущем треке, а также расширение форматов воспроизводимых аудиофайлов.

Таким образом, выполненная работа позволила закрепить знания по разработке мобильных приложений на Kotlin и продемонстрировала базовые навыки создания мультимедийных приложений на платформе Android.

\endinput
