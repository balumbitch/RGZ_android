\chapter{Практическая часть}

\section{Создание интерфейса пользователя}

В приложении реализован простой и интуитивно понятный интерфейс, состоящий из следующих компонентов:

\begin{itemize}
    \item \textbf{TextView} для отображения текущего трека.
    \item \textbf{SeekBar} для визуального отображения и управления прогрессом воспроизведения.
    \item \textbf{Button} для управления воспроизведением: Play/Pause, Next, Back, Cycle.
    \item \textbf{Button} для управления громкостью: Volume Up, Volume Down.
\end{itemize}

Все компоненты интерфейса размещены в XML-файле \texttt{activity\_mp3.xml}. Вёрстка выполнена с использованием ConstraintLayout для гибкости и адаптивности интерфейса.

\section{Загрузка и обработка музыкальных файлов}

В проекте реализовано два метода для загрузки музыкальных файлов:

\begin{enumerate}
    \item \textbf{Из хранилища устройства:}  
    Используется \texttt{ContentResolver} для сканирования медиафайлов в директории \texttt{/Music}. Файлы фильтруются по расширению \texttt{.mp3}, после чего их пути сохраняются в списке \texttt{songs}.

    \item \textbf{Из директории /Music:}  
    Реализована проверка на существование директории \texttt{/Music} и считывание файлов напрямую через \texttt{File} API.
\end{enumerate}

Пример загрузки файла из медиа-хранилища:

\begin{lstlisting}[language=Kotlin, caption={Загрузка музыки из MediaStore}]
private fun loadMusicFromMediaStore() {
    val uri: Uri = MediaStore.Audio.Media.EXTERNAL_CONTENT_URI
    val projection = arrayOf(
        MediaStore.Audio.Media.DATA,
        MediaStore.Audio.Media.TITLE
    )
    val selection = MediaStore.Audio.Media.IS_MUSIC + "!= 0"
    
    contentResolver.query(uri, projection, selection, null, null)?.use { cursor ->
        while (cursor.moveToNext()) {
            val path = cursor.getString(0)
            val title = cursor.getString(1)
            songs.add(path)
            songTitles.add(title)
        }
    }
}
\end{lstlisting}

\section{Инициализация и воспроизведение аудио}

Для воспроизведения MP3 используется класс \texttt{MediaPlayer}. В процессе инициализации создаётся новый экземпляр \texttt{MediaPlayer} и устанавливается путь к текущему треку:

\begin{lstlisting}[language=Kotlin, caption={Инициализация MediaPlayer}]
private fun initializeMediaPlayer() {
    if (songs.isNotEmpty()) {
        mediaPlayer = MediaPlayer()
        mediaPlayer.setDataSource(songs[currentSongIndex])
        mediaPlayer.prepare()
        updateSongTitle()
        setupSeekBar()
        setupButtons()
    }
}
\end{lstlisting}

\section{Реализация управления воспроизведением}

Основные функции управления воспроизведением:

\begin{itemize}
\item \textbf{Play/Pause:} Переключает состояние воспроизведения и обновляет текст кнопки.
\item \textbf{Next/Back:} Изменяет индекс текущего трека и перезапускает воспроизведение.
\item \textbf{Cycle:} Включает/выключает режим цикличного воспроизведения.
\item \textbf{SeekBar:} Отслеживает текущую позицию трека и позволяет перематывать воспроизведение.
\end{itemize}

Пример реализации кнопки \texttt{Play/Pause}:

\begin{lstlisting}[language=Kotlin, caption={Обработка нажатия кнопки Play/Pause}]
play.setOnClickListener {
    if (!mediaPlayer.isPlaying) {
        mediaPlayer.start()
        play.text = "Pause"
    } else {
        mediaPlayer.pause()
        play.text = "Play"
    }
}
\end{lstlisting}

\section{Обработка разрешений}

Для доступа к музыкальным файлам приложение запрашивает разрешение на чтение внешнего хранилища:

\begin{lstlisting}[language=Kotlin, caption={Проверка разрешений}]
private fun checkPermission() {
    val permission = if (Build.VERSION.SDK_INT >= Build.VERSION_CODES.TIRAMISU) {
        Manifest.permission.READ_MEDIA_AUDIO
    } else {
        Manifest.permission.READ_EXTERNAL_STORAGE
    }

    when {
        ContextCompat.checkSelfPermission(this, permission) == PackageManager.PERMISSION_GRANTED -> {
            loadMusic()
        }
        else -> {
            requestPermissionLauncher.launch(permission)
        }
    }
}
\end{lstlisting}

\section{Тестирование и отладка}

В ходе тестирования было проверено:

\begin{itemize}
\item Корректность загрузки музыкальных файлов из \texttt{/Music}.
\item Корректная работа кнопок управления воспроизведением.
\item Отображение текущего трека и его позиции на \texttt{SeekBar}.
\item Обработка ошибок при отсутствии файлов или отсутствии разрешений.
\end{itemize}

Таким образом, приложение реализует все основные функции MP3 плеера, включая воспроизведение, переключение треков, регулировку громкости и цикличное воспроизведение.
