\chapter{Теоретическая часть}

\section{Формат MP3 и его особенности}

MP3 — это формат аудиосжатия с потерями, который позволяет уменьшить размер звуковых файлов, сохраняя приемлемое качество воспроизведения. Сжатие осуществляется за счёт удаления избыточных данных, которые человеческое ухо практически не воспринимает.  

Для воспроизведения MP3 файлов требуется декодер, который преобразует сжатый поток данных в PCM (Pulse Code Modulation) формат, пригодный для воспроизведения через динамики.

\section{Основные компоненты MP3 плеера на Kotlin}

MP3 плеер реализован на языке Kotlin и включает следующие основные компоненты:

\begin{itemize}
    \item \textbf{MediaPlayer} — встроенный класс Android SDK для воспроизведения аудио.
    \item \textbf{AudioManager} — для управления громкостью и аудиопотоками.
    \item \textbf{Handler} — для обновления позиции воспроизведения и синхронизации с SeekBar.
    \item \textbf{ContentResolver} — для доступа к музыкальным файлам, хранящимся на устройстве.
\end{itemize}

\section{Архитектура приложения}

Приложение представляет собой одноактивити с XML-интерфейсом и функциональными кнопками управления:

\begin{itemize}
    \item \textbf{Play/Pause} — старт и пауза воспроизведения.
    \item \textbf{Next/Back} — переключение между треками.
    \item \textbf{Cycle} — включение и отключение цикличного воспроизведения.
    \item \textbf{Volume Up/Down} — управление уровнем громкости через AudioManager.
\end{itemize}

Для организации воспроизведения MP3 файлов используется локальный список песен, формируемый из директории \texttt{/Music} и медиа-хранилища устройства.

\section{Обработка разрешений}

Для доступа к музыкальным файлам приложение запрашивает разрешение на чтение данных. В зависимости от версии Android это может быть \texttt{READ\_EXTERNAL\_STORAGE} или \texttt{READ\_MEDIA\_AUDIO}.  

Система реализована через \texttt{ActivityResultContracts.RequestPermission()}, что позволяет гибко управлять процессом запроса разрешений.

\section{Обработка событий и состояния приложения}

В приложении реализована система управления состояниями:

\begin{itemize}
    \item \textbf{Инициализация и подготовка} — загрузка треков и настройка MediaPlayer.
    \item \textbf{Воспроизведение и пауза} — обновление кнопки Play/Pause в зависимости от состояния MediaPlayer.
    \item \textbf{Переключение треков} — корректное обновление индекса трека и перезапуск воспроизведения.
    \item \textbf{Завершение воспроизведения} — автоматический переход к следующему треку или возврат к первому треку в списке.
\end{itemize}

Таким образом, теоретическая часть охватывает основные компоненты и архитектуру MP3 плеера, реализованного на Kotlin с использованием Android SDK.

