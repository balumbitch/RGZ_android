% !TeX root = 0_main.tex
% !TeX program = xelatex
\documentclass[a4paper,14pt,oneside,openany]{memoir}

%%% Задаем поля, отступы и межстрочный интервал %%%

\usepackage[left=30mm, right=15mm, top=20mm, bottom=20mm]{geometry} % Пакет geometry с аргументами для определения полей
\pagestyle{plain} % Убираем стандарные для данного класса верхние колонтитулы с заголовком текущей главы, оставляем только номер страницы снизу по центру
\parindent=1.25cm % Абзацный отступ 1.25 см, приблизительно равно пяти знакам, как по ГОСТ
\usepackage{indentfirst} % Добавляем отступ к первому абзацу
%\linespread{1.3} % Межстрочный интервал (наиболее близко к вордовскому полуторному) - тут вместо этого используется команда OnehalfSpacing*

%%% Задаем языковые параметры и шрифт %%%

\usepackage[english, russian]{babel}                % Настройки для русского языка как основного в тексте
\babelfont[russian]{rm}{Times New Roman}                     % TMR в качестве базового roman-щрифта



%%% Задаем стиль заголовков и подзаголовков в тексте %%%

\setsecnumdepth{subsection} % Номера разделов считать до третьего уровня включительно, т.е. нумеруются только главы, секции, подсекции
\renewcommand*{\chapterheadstart}{} % Переопределяем команду, задающую отступ над заголовком, чтобы отступа не было
\renewcommand*{\printchaptername}{} % Переопределяем команду, печатающую слово "Глава", чтобы оно не печалось
%\renewcommand*{\printchapternum}{} % То же самое для номера главы - тут не надо, номер главы оставляем
\renewcommand*{\chapnumfont}{\normalfont\bfseries} % Меняем стиль шрифта для номера главы: нормальный размер, полужирный
\renewcommand*{\afterchapternum}{\hspace{1em}} % Меняем разделитель между номером главы и названием
\renewcommand*{\printchaptertitle}{\normalfont\bfseries\centering\MakeUppercase} % Меняем стиль написания для заголовка главы: нормальный размер, полужирный, центрированный, заглавными буквами
\setbeforesecskip{20pt} % Задаем отступ перед заголовком секции
\setaftersecskip{20pt} % Ставим такой же отступ после заголовка секции
\setsecheadstyle{\raggedright\normalfont\bfseries} % Меняем стиль написания для заголовка секции: выравнивание по правому краю без переносов, нормальный размер, полужирный
\setbeforesubsecskip{20pt} % Задаем отступ перед заголовком подсекции
\setaftersubsecskip{20pt} % Ставим такой же отступ после заголовка подсекции
\setsubsecheadstyle{\raggedright\normalfont\bfseries}  % Меняем стиль написания для заголовка подсекции: выравнивание по правому краю без переносов, нормальный размер, полужирный

%%% Задаем параметры оглавления %%%

\addto\captionsrussian{\renewcommand\contentsname{Содержание}} % Меняем слово "Оглавление" на "Содержание"
\setrmarg{2.55em plus1fil} % Запрещаем переносы слов в оглавлении
%\setlength{\cftbeforechapterskip}{0pt} % Эта команда убирает интервал между заголовками глав - тут не надо, так красивее смотрится
\renewcommand{\aftertoctitle}{\afterchaptertitle \vspace{-\cftbeforechapterskip}} % Делаем отступ между словом "Содержание" и первой строкой таким же, как у заголовков глав
%\renewcommand*{\chapternumberline}[1]{} % Делаем так, чтобы номер главы не печатался - тут не надо
\renewcommand*{\cftchapternumwidth}{1.5em} % Ставим подходящий по размеру разделитель между номером главы и самим заголовком
\renewcommand*{\cftchapterfont}{\normalfont\MakeUppercase} % Названия глав обычным шрифтом заглавными буквами
\renewcommand*{\cftchapterpagefont}{\normalfont} % Номера страниц обычным шрифтом
\renewcommand*{\cftchapterdotsep}{\cftdotsep} % Делаем точки до номера страницы после названий глав
\renewcommand*{\cftdotsep}{1} % Задаем расстояние между точками
\renewcommand*{\cftchapterleader}{\cftdotfill{\cftchapterdotsep}} % Делаем точки стандартной формы (по умолчанию они "жирные")
\maxtocdepth{subsection} % В оглавление попадают только разделы первыхтрех уровней: главы, секции и подсекции

%%% Выравнивание и переносы %%%

%% http://tex.stackexchange.com/questions/241343/what-is-the-meaning-of-fussy-sloppy-emergencystretch-tolerance-hbadness
%% http://www.latex-community.org/forum/viewtopic.php?p=70342#p70342
\tolerance 1414
\hbadness 1414
\emergencystretch 1.5em                             % В случае проблем регулировать в первую очередь
\hfuzz 0.3pt
\vfuzz \hfuzz
%\dbottom
%\sloppy                                            % Избавляемся от переполнений
\clubpenalty=10000                                  % Запрещаем разрыв страницы после первой строки абзаца
\widowpenalty=10000                                 % Запрещаем разрыв страницы после последней строки абзаца
\brokenpenalty=4991                                 % Ограничение на разрыв страницы, если строка заканчивается переносом

%%% Объясняем компилятору, какие буквы русского алфавита можно использовать в перечислениях (подрисунках и нумерованных списках) %%%
%%% По ГОСТ нельзя использовать буквы ё, з, й, о, ч, ь, ы, ъ %%%
%%% Здесь также переопределены заглавные буквы, хотя в принципе они в документе не используются %%%

\makeatletter
    \def\russian@Alph#1{\ifcase#1\or
       А\or Б\or В\or Г\or Д\or Е\or Ж\or
       И\or К\or Л\or М\or Н\or
       П\or Р\or С\or Т\or У\or Ф\or Х\or
       Ц\or Ш\or Щ\or Э\or Ю\or Я\else\xpg@ill@value{#1}{russian@Alph}\fi}
    \def\russian@alph#1{\ifcase#1\or
       а\or б\or в\or г\or д\or е\or ж\or
       и\or к\or л\or м\or н\or
       п\or р\or с\or т\or у\or ф\or х\or
       ц\or ш\or щ\or э\or ю\or я\else\xpg@ill@value{#1}{russian@alph}\fi}
\makeatother

%%% Задаем параметры оформления рисунков и таблиц %%%

\usepackage{graphicx, caption, subcaption} % Подгружаем пакеты для работы с графикой и настройки подписей
\graphicspath{{images/}} % Определяем папку с рисунками
\captionsetup[figure]{font=small, width=\textwidth, name=Рисунок, justification=centering} % Задаем параметры подписей к рисункам: маленький шрифт (в данном случае 12pt), ширина равна ширине текста, полнотекстовая надпись "Рисунок", выравнивание по центру
\captionsetup[subfigure]{font=small} % Индексы подрисунков а), б) и так далее тоже шрифтом 12pt (по умолчанию делает еще меньше)
\captionsetup[table]{singlelinecheck=false,font=small,width=\textwidth,justification=justified} % Задаем параметры подписей к таблицам: запрещаем переносы, маленький шрифт (в данном случае 12pt), ширина равна ширине текста, выравнивание по ширине
\captiondelim{ --- } % Разделителем между номером рисунка/таблицы и текстом в подписи является длинное тире
\setkeys{Gin}{width=\textwidth} % По умолчанию размер всех добавляемых рисунков будет подгоняться под ширину текста
\renewcommand{\thesubfigure}{\asbuk{subfigure}} % Нумерация подрисунков строчными буквами кириллицы
%\setlength{\abovecaptionskip}{0pt} % Отбивка над подписью - тут не меняем
%\setlength{\belowcaptionskip}{0pt} % Отбивка под подписью - тут не меняем
\usepackage[section]{placeins} % Объекты типа float (рисунки/таблицы) не вылезают за границы секциии, в которой они объявлены

%%% Задаем параметры ссылок и гиперссылок %%% 

\usepackage{hyperref}                               % Подгружаем нужный пакет
\hypersetup{
    colorlinks=true,                                % Все ссылки и гиперссылки цветные
    linktoc=all,                                    % В оглавлении ссылки подключатся для всех отображаемых уровней
    linktocpage=true,                               % Ссылка - только номер страницы, а не весь заголовок (так выглядит аккуратнее)
    linkcolor=red,                                  % Цвет ссылок и гиперссылок - красный
    citecolor=red                                   % Цвет цитировний - красный
}

%%% Настраиваем отображение списков %%%

\usepackage{enumitem}                               % Подгружаем пакет для гибкой настройки списков
\renewcommand*{\labelitemi}{\normalfont{--}}        % В ненумерованных списках для пунктов используем короткое тире
\makeatletter
    \AddEnumerateCounter{\asbuk}{\russian@alph}     % Объясняем пакету enumitem, как использовать asbuk
\makeatother
\renewcommand{\labelenumii}{\asbuk{enumii}}        % Кириллица для второго уровня нумерации
\renewcommand{\labelenumiii}{\arabic{enumiii}}     % Арабские цифры для третьего уровня нумерации
\setlist{noitemsep, leftmargin=*}                   % Убираем интервалы между пунками одного уровня в списке
\setlist[1]{labelindent=\parindent}                 % Отступ у пунктов списка равен абзацному отступу
\setlist[2]{leftmargin=\parindent}                  % Плюс еще один такой же отступ для следующего уровня
\setlist[3]{leftmargin=\parindent}                  % И еще один для третьего уровня

%%% Счетчики для нумерации объектов %%%

\counterwithout{figure}{chapter}                    % Сквозная нумерация рисунков по документу
\counterwithout{equation}{chapter}                  % Сквозная нумерация математических выражений по документу
\counterwithout{table}{chapter}                     % Сквозная нумерация таблиц по документу

%%% Реализация библиографии пакетами biblatex и biblatex-gost с использованием движка biber %%%

\usepackage{csquotes} % Пакет для оформления сложных блоков цитирования (biblatex рекомендует его подключать)
\usepackage[%
backend=biber,                                      % Движок
bibencoding=utf8,                                   % Кодировка bib-файла
sorting=none,                                       % Настройка сортировки списка литературы
style=gost-numeric,                                 % Стиль цитирования и библиографии по ГОСТ
language=auto,                                      % Язык для каждой библиографической записи задается отдельно
autolang=other,                                     % Поддержка многоязычной библиографии
sortcites=true,                                     % Если в квадратных скобках несколько ссылок, то отображаться будут отсортированно
movenames=false,                                    % Не перемещать имена, они всегда в начале библиографической записи
maxnames=5,                                         % Максимальное отображаемое число авторов
minnames=3,                                         % До скольки сокращать число авторов, если их больше максимума
doi=false,                                          % Не отображать ссылки на DOI
isbn=false,                                         % Не показывать ISBN, ISSN, ISRN
]{biblatex}[2016/09/17]
\DeclareDelimFormat{bibinitdelim}{}                 % Убираем пробел между инициалами (Иванов И.И. вместо Иванов И. И.)
\addbibresource{bibl.bib}                           % Определяем файл с библиографией

%%% Скрипт, который автоматически подбирает язык (и, следовательно, формат) для каждой библиографической записи %%%
%%% Если в названии работы есть кириллица - меняем значение поля langid на russian %%%
%%% Все оставшиеся пустые места в поле langid заменяем на english %%%

\DeclareSourcemap{
  \maps[datatype=bibtex]{
    \map{
        \step[fieldsource=title, match=\regexp{^\P{Cyrillic}*\p{Cyrillic}.*}, final]
        \step[fieldset=langid, fieldvalue={russian}]
    }
    \map{
        \step[fieldset=langid, fieldvalue={english}]
    }
  }
}

%%% Прочие пакеты для расширения функционала %%%

\usepackage{longtable,ltcaption}                    % Длинные таблицы
\usepackage{multirow,makecell}                      % Улучшенное форматирование таблиц
\usepackage{booktabs}                               % Еще один пакет для красивых таблиц
\usepackage{soulutf8}                               % Поддержка переносоустойчивых подчёркиваний и зачёркиваний
\usepackage{icomma}                                 % Запятая в десятичных дробях
\usepackage{hyphenat}                               % Для красивых переносов
\usepackage{textcomp}                               % Поддержка "сложных" печатных символов типа значков иены, копирайта и т.д.
\usepackage[version=4]{mhchem}                      % Красивые химические уравнения
\usepackage{amsmath}                                % Усовершенствование отображения математических выражений 
\usepackage{listings}
\usepackage{xcolor}


\lstdefinelanguage{Kotlin}{
    keywords={package, import, class, fun, val, var, object, override, if, else, for, while, do, return, when, try, catch},
    sensitive=true,
    comment=[l]{//},
    morecomment=[s]{/*}{*/},
    morestring=[b]",
    morestring=[b]',
    keywordstyle=\color{blue}\bfseries,
    commentstyle=\color{gray}\itshape,
    stringstyle=\color{green!70!black},
}

\lstset{
    basicstyle=\ttfamily\small,
    numbers=left,
    numberstyle=\tiny\color{gray},
    breaklines=true,
    breakatwhitespace=true,
    frame=single,
    framerule=0.5pt,
    rulecolor=\color{black!20},
    captionpos=b,
    tabsize=4,
    showspaces=false,
    showstringspaces=false,
    showtabs=false
}
%%% Вставляем по очереди все содержательные части документа %%%

\begin{document}

\thispagestyle{empty}

\begin{center}
    МИНИСТЕРСТВО ЦИФРОВОГО РАЗВИТИЯ, СВЯЗИ И МАССОВЫХ КОММУНИКАЦИЙ \\ РОССИЙСКОЙ ФЕДЕРАЦИИ

    \vspace{20pt}

    Федеральное государственное бюджетное образовательное учреждение  \\  высшего образования \\
    "<Сибирский государственный университет телекоммуникаций и информатики"> \\

    \vspace{20pt}

    Кафедра телекоммуникационных систем и вычислительных средств \\  (ТС и ВС)
\end{center}

\vfill

\begin{center}
    Расчетно-графическая работа \\  
    по дисциплине \\
    \textit{"<Визуальное программирование">}

    \vspace{20pt}

    по теме: \\
    \uppercase{MP3 Player}
\end{center}

\vfill

    \noindent Студент: \\
    \textit{Группа ИА-332 \hfill В.П. Шепталин}

    \vspace{20pt}

    \noindent Преподаватель: \hfill \textit{Р.В. Ахпашев}


\vfill

\begin{center}
    Новосибирск 2025 г.
\end{center}                                     % Титульник

\newpage % Переходим на новую страницу
\setcounter{page}{2} % Начинаем считать номера страниц со второй
\OnehalfSpacing* % Задаем полуторный интервал текста (в титульнике одинарный, поэтому команда стоит после него)

\tableofcontents*                                   % Автособираемое оглавление

\chapter*{Введение}
\addcontentsline{toc}{chapter}{Введение}
\label{ch:intro}

В современном мире мобильные устройства и программное обеспечение для воспроизведения мультимедиа играют важную роль в повседневной жизни. MP3-плееры — это одно из самых популярных средств для прослушивания аудиофайлов различных форматов. Цель данной работы — разработка простого и функционального MP3-плеера на языке программирования Kotlin, который широко используется для создания приложений под платформу Android.

В ходе выполнения работы рассматриваются основные принципы воспроизведения аудио, управление воспроизведением (воспроизведение, пауза, остановка, переход между треками), а также организация пользовательского интерфейса с помощью современных средств разработки Android-приложений. Особое внимание уделяется интеграции с системными ресурсами устройства и обеспечению удобства использования.

Данный проект позволяет закрепить практические навыки программирования на Kotlin, познакомиться с особенностями работы с аудиоданными и интерфейсом Android, а также получить опыт реализации мультимедийных приложений, востребованных в современных технологиях.
                                     % Введение
\chapter{Теоретическая часть}

\section{Формат MP3 и его особенности}

MP3 — это формат аудиосжатия с потерями, который позволяет уменьшить размер звуковых файлов, сохраняя приемлемое качество воспроизведения. Сжатие осуществляется за счёт удаления избыточных данных, которые человеческое ухо практически не воспринимает.  

Для воспроизведения MP3 файлов требуется декодер, который преобразует сжатый поток данных в PCM (Pulse Code Modulation) формат, пригодный для воспроизведения через динамики.

\section{Основные компоненты MP3 плеера на Kotlin}

MP3 плеер реализован на языке Kotlin и включает следующие основные компоненты:

\begin{itemize}
    \item \textbf{MediaPlayer} — встроенный класс Android SDK для воспроизведения аудио.
    \item \textbf{AudioManager} — для управления громкостью и аудиопотоками.
    \item \textbf{Handler} — для обновления позиции воспроизведения и синхронизации с SeekBar.
    \item \textbf{ContentResolver} — для доступа к музыкальным файлам, хранящимся на устройстве.
\end{itemize}

\section{Архитектура приложения}

Приложение представляет собой одноактивити с XML-интерфейсом и функциональными кнопками управления:

\begin{itemize}
    \item \textbf{Play/Pause} — старт и пауза воспроизведения.
    \item \textbf{Next/Back} — переключение между треками.
    \item \textbf{Cycle} — включение и отключение цикличного воспроизведения.
    \item \textbf{Volume Up/Down} — управление уровнем громкости через AudioManager.
\end{itemize}

Для организации воспроизведения MP3 файлов используется локальный список песен, формируемый из директории \texttt{/Music} и медиа-хранилища устройства.

\section{Обработка разрешений}

Для доступа к музыкальным файлам приложение запрашивает разрешение на чтение данных. В зависимости от версии Android это может быть \texttt{READ\_EXTERNAL\_STORAGE} или \texttt{READ\_MEDIA\_AUDIO}.  

Система реализована через \texttt{ActivityResultContracts.RequestPermission()}, что позволяет гибко управлять процессом запроса разрешений.

\section{Обработка событий и состояния приложения}

В приложении реализована система управления состояниями:

\begin{itemize}
    \item \textbf{Инициализация и подготовка} — загрузка треков и настройка MediaPlayer.
    \item \textbf{Воспроизведение и пауза} — обновление кнопки Play/Pause в зависимости от состояния MediaPlayer.
    \item \textbf{Переключение треков} — корректное обновление индекса трека и перезапуск воспроизведения.
    \item \textbf{Завершение воспроизведения} — автоматический переход к следующему треку или возврат к первому треку в списке.
\end{itemize}

Таким образом, теоретическая часть охватывает основные компоненты и архитектуру MP3 плеера, реализованного на Kotlin с использованием Android SDK.

                                     % Первая глава
\chapter{Практическая часть}

\section{Создание интерфейса пользователя}

В приложении реализован простой и интуитивно понятный интерфейс, состоящий из следующих компонентов:

\begin{itemize}
    \item \textbf{TextView} для отображения текущего трека.
    \item \textbf{SeekBar} для визуального отображения и управления прогрессом воспроизведения.
    \item \textbf{Button} для управления воспроизведением: Play/Pause, Next, Back, Cycle.
    \item \textbf{Button} для управления громкостью: Volume Up, Volume Down.
\end{itemize}

Все компоненты интерфейса размещены в XML-файле \texttt{activity\_mp3.xml}. Вёрстка выполнена с использованием ConstraintLayout для гибкости и адаптивности интерфейса.

\section{Загрузка и обработка музыкальных файлов}

В проекте реализовано два метода для загрузки музыкальных файлов:

\begin{enumerate}
    \item \textbf{Из хранилища устройства:}  
    Используется \texttt{ContentResolver} для сканирования медиафайлов в директории \texttt{/Music}. Файлы фильтруются по расширению \texttt{.mp3}, после чего их пути сохраняются в списке \texttt{songs}.

    \item \textbf{Из директории /Music:}  
    Реализована проверка на существование директории \texttt{/Music} и считывание файлов напрямую через \texttt{File} API.
\end{enumerate}

Пример загрузки файла из медиа-хранилища:

\begin{lstlisting}[language=Kotlin, caption={Загрузка музыки из MediaStore}]
private fun loadMusicFromMediaStore() {
    val uri: Uri = MediaStore.Audio.Media.EXTERNAL_CONTENT_URI
    val projection = arrayOf(
        MediaStore.Audio.Media.DATA,
        MediaStore.Audio.Media.TITLE
    )
    val selection = MediaStore.Audio.Media.IS_MUSIC + "!= 0"
    
    contentResolver.query(uri, projection, selection, null, null)?.use { cursor ->
        while (cursor.moveToNext()) {
            val path = cursor.getString(0)
            val title = cursor.getString(1)
            songs.add(path)
            songTitles.add(title)
        }
    }
}
\end{lstlisting}

\section{Инициализация и воспроизведение аудио}

Для воспроизведения MP3 используется класс \texttt{MediaPlayer}. В процессе инициализации создаётся новый экземпляр \texttt{MediaPlayer} и устанавливается путь к текущему треку:

\begin{lstlisting}[language=Kotlin, caption={Инициализация MediaPlayer}]
private fun initializeMediaPlayer() {
    if (songs.isNotEmpty()) {
        mediaPlayer = MediaPlayer()
        mediaPlayer.setDataSource(songs[currentSongIndex])
        mediaPlayer.prepare()
        updateSongTitle()
        setupSeekBar()
        setupButtons()
    }
}
\end{lstlisting}

\section{Реализация управления воспроизведением}

Основные функции управления воспроизведением:

\begin{itemize}
\item \textbf{Play/Pause:} Переключает состояние воспроизведения и обновляет текст кнопки.
\item \textbf{Next/Back:} Изменяет индекс текущего трека и перезапускает воспроизведение.
\item \textbf{Cycle:} Включает/выключает режим цикличного воспроизведения.
\item \textbf{SeekBar:} Отслеживает текущую позицию трека и позволяет перематывать воспроизведение.
\end{itemize}

Пример реализации кнопки \texttt{Play/Pause}:

\begin{lstlisting}[language=Kotlin, caption={Обработка нажатия кнопки Play/Pause}]
play.setOnClickListener {
    if (!mediaPlayer.isPlaying) {
        mediaPlayer.start()
        play.text = "Pause"
    } else {
        mediaPlayer.pause()
        play.text = "Play"
    }
}
\end{lstlisting}

\section{Обработка разрешений}

Для доступа к музыкальным файлам приложение запрашивает разрешение на чтение внешнего хранилища:

\begin{lstlisting}[language=Kotlin, caption={Проверка разрешений}]
private fun checkPermission() {
    val permission = if (Build.VERSION.SDK_INT >= Build.VERSION_CODES.TIRAMISU) {
        Manifest.permission.READ_MEDIA_AUDIO
    } else {
        Manifest.permission.READ_EXTERNAL_STORAGE
    }

    when {
        ContextCompat.checkSelfPermission(this, permission) == PackageManager.PERMISSION_GRANTED -> {
            loadMusic()
        }
        else -> {
            requestPermissionLauncher.launch(permission)
        }
    }
}
\end{lstlisting}

\section{Тестирование и отладка}

В ходе тестирования было проверено:

\begin{itemize}
\item Корректность загрузки музыкальных файлов из \texttt{/Music}.
\item Корректная работа кнопок управления воспроизведением.
\item Отображение текущего трека и его позиции на \texttt{SeekBar}.
\item Обработка ошибок при отсутствии файлов или отсутствии разрешений.
\end{itemize}

Таким образом, приложение реализует все основные функции MP3 плеера, включая воспроизведение, переключение треков, регулировку громкости и цикличное воспроизведение.
                                     % Вторая глава
\chapter*{Заключение}
\addcontentsline{toc}{chapter}{Заключение}

В рамках данной расчетно-графической работы было разработано мобильное приложение MP3-плеер на языке программирования Kotlin с использованием фреймворка Android. Основной функционал приложения включает в себя воспроизведение музыкальных файлов из хранилища устройства, управление треками (воспроизведение, пауза, переход к следующему и предыдущему треку), регулировку громкости и активацию режима цикличного воспроизведения.

Приложение успешно реализует задачи, поставленные в техническом задании:  
1. Загрузка музыкальных файлов из директории \texttt{/Music} и медиа-хранилища устройства.  
2. Управление воспроизведением с помощью кнопок Play/Pause, Next и Back.  
3. Отображение информации о текущем треке и его состоянии на SeekBar.  
4. Реализация простого и интуитивно понятного интерфейса на основе ConstraintLayout.  
5. Обработка системных разрешений для доступа к аудиофайлам.  

В ходе выполнения работы изучены основные компоненты Android SDK для работы с мультимедиа, такие как \texttt{MediaPlayer} и \texttt{AudioManager}. Также были применены базовые подходы к организации пользовательского интерфейса и управления разрешениями.

Основным направлением для дальнейшего улучшения приложения может стать реализация плейлиста с возможностью выбора треков пользователем, создание уведомлений о текущем треке, а также расширение форматов воспроизводимых аудиофайлов.

Таким образом, выполненная работа позволила закрепить знания по разработке мобильных приложений на Kotlin и продемонстрировала базовые навыки создания мультимедийных приложений на платформе Android.

\endinput
                                     % Третья глава

\nocite{*}
\printbibliography[title=Список использованных источников] % Автособираемый список литературы

\end{document}